\section{HexEditor}\label{sec:hexeditor}

Arten wie eine Datei im HexEditor innerhalb \codeblocks geöffnet werden kann.

\begin{enumerate}
\item \menu{File, Open with HexEditor}
\item Kontextmenü des Projekt Navigator (\menu{Open with, Hex editor}
\item Wählen Sie den Reiter Tab Files im Management Panel aus. Durch Auswahl einer Datei im FileManager und Ausführen des Kontextmenü \menu{Open With Hex editor} wird diese Datei im HexEditor geöffnet.
\end{enumerate}

Aufteilung Fenster:

links HexEditor Ansicht und rechts die Anzeige als Strings
Obere Zeile:
Aktuelle Position (Wert in Dezimal/Hex) und Prozentzahl (Verhältnis der aktuellen Cursor Position zur Gesamtdatei).

Buttons:

Suchfunktionen

Goto Button: Sprung an eine absolute Position. Format in Dezimal oder Hex. Relativer Sprung vorwärts oder rückwarts durch Angabe eines Vorzeichens.
Search: Suche von Hex-Pattern in der HexEditor Ansicht bzw. auch nach Strings in der Dateivorschau Ansicht.

Konfiguration der Spaltenzahl:
Exactly, Multiple of, Power of

Anzeigemodus:
Hex, Binary

Bytes:
Auswahl wieviel Bytes pro Spalte angezeigt werden sollen.

Wahl der Endianess:
BE: Big Endian
LE: Little Endian

Value Preview:
Fügt eine zusätzliche Ansicht im HexEditor hinzu. Für ein selektierten Wert im HexEditor wird der Wert auch als Word, Dword, Float, Dobule angezeigt werde.

Expression Eingabe:
Ermöglicht eine Rechenoperation auf eine Wert im HexEditor auszuführen. Ergebnis der Operation wird am rechten Rand angezeigt.

Calc:
Expression Tester

Bearbeiten einer Datei im HexEditor:

Verfügt Undo und Redo History.
Z.B. Cursor in die Stringansicht bewegen:
Einfügen von Leerzeichen mit Einfg Taste.
Löschen von Zeichen mit Taste Entf starten.

Durch Eingabe eines Textes wird der bestehende Inhalt als String überschrieben.

Durch Eingabe von Zahlen in der HexEditor Ansicht werden die Werte überschrieben und die Vorschau aktualisiert.

